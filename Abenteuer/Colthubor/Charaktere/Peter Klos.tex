\subsection{Peter Klos}
    \label{pers:peter-klos}
    \begin{minipage}{0.7\textwidth}
        Diese Person ist eine Art Aufpasser in der \nameref{locs:bibliothek-der-alten-schriften}. Er ist ein recht altes und sehr radikales Mitglied des \hyperref[sons:kult-von-colthubor]{Kultes von Colthubor} und sieht es entsprechend ungern, wenn \emph{Nichtmitglieder} in der \hyperref[locs:bibliothek-der-alten-schriften]{Bibliothek} \emph{rumschnüffeln}.
    \end{minipage}
    \begin{minipage}{0.3\textwidth}
        \centering
        \begin{tabular}{| l | l |}
            \hline
            Lügen erkennen & $25$ \\ \hline
            Faustkampf & $30$ \\ \hline
            Schwertkampf & $60$ \\ \hline
        \end{tabular}
    \end{minipage}

    \subsubsection{Erscheinungsbild}
        Wie alle Anhänge in der \nameref{locs:bibliothek-der-alten-schriften}, trägt auch \emph{Peter Klos} eine dunklen Hose mit einem Gewand darüber, welches einem Kleid ähnelt, welches an der Seite bis zu Hüfte aufgeschnitten ist. Er ist etwa $50$ Jahre alt und trägt einen Dreitagebart. Das Gesicht wirkt rustikal, aber nicht ungepflegt. Mit einer Größe von etwa $190$ cm und dank der sehr lockeren Kleidung, ist die genau Statur nicht klar erkennbar.

    \subsubsection{Verhalten}
        Weil es sich bei \emph{Peter Klos} um einen Sadisten handelt, der die \nameref{sons:kult-von-colthubor:alte-schriften} als das erkannt hat, was sie sind, eine Anleitung die Welt ins Chaos und in die Verderbnis zu stürzen, ist er ein sehr misstrauischer Mensch.

        Grundsätzlich lässt \emph{Peter Klos} auch nicht mit sich verhandeln und wenn er merkt, dass er belogen oder hinterslicht geführt wird, so ist er stets ein Freund des \emph{erst schlagen, dann fragen} gewesen.

    \subsubsection{Wissen}
        \emph{Peter Klos} gilt als Leser der jüngsten Stunde in den Bibliothek. Er weiß nicht nur was alles niedergeschrieben ist, sondern auch wo die entsprechenden Schriften zu finden sind. Über diese Redet er jedoch nicht bereitwillig, er müsste schon verbal ausgetrickst werden.

    \paragraph{Inventar:} Kurzschwert

    \paragraph{Mögliche Erscheingungsorge:} \nameref{locs:bibliothek-der-alten-schriften}
