\subsection{George Smith}
\label{pers:george-smith}
    \emph{George Smith} ist nicht nur ein praktizierendes Mitglied des \hyperref[sons:kult-von-colthubor]{Kultes von Colthubor}, sondern oberdrein noch ein hochrangiges, welches Rituale leitet. Zu Beginn des Abenteuers befindet sich \emph{George Smith} im Gefängnis, da eine Zeremonie unter seiner Leitung, in welchem es unter anderem menschliche Blutopfer gab, von der \hyperref[subsubsec:billo]{BiLLo} gefangen genommen. Dies war nur möglich, da die \hyperref[subsubsec:billo]{BiLLo} einen Spitzel eingeschleust hatte.

    \subsubsection{Erscheinungsbild}
        Als Teil seines Glaubens, hat sich \emph{George Smith} tätowieren lassen. Seinen ganzen Körper zieren auf den ersten oder ungeschulten Blick wirre Schnörkelein und mit einigen Kreisen, mit Ausnahme von Händen, Füßen und dem Kopf. Sonst handelt es sich um einen groß gewachsenen aber hargeren Mann. Diese bewegt sich an sich in seiner etwas mitgenommenen und recht spärlichen Gefängnisbekleidung relativ geschmeidig, hält jedoch seine linke Schulter fast durchgehend mit der rechten Hand fest.Auf die Aufforderung die Schulter einmal los zulassen, behauptet \emph{George Smith} die Schulter sei verletzt und so sei es nicht so schmerzhaft.

    \subsubsection{Verhalten}
        Grundsätzlich ist \emph{George Smith} ein sehr ruhiger, selbstsicherer und aroganter Mensch. Im Gefängnis verhöhnt er die Protagonisten, da diese die Dinge die da harren ohne nicht nicht aufhalten könnten. Dies, im Zusammenhang mit der Aroganz, lässt ihn schwadronierend über fast alles reden, was die Protagonisten wissen wollen. Über die gewöhnliche Kooperation hinaus, bietet er den Protagonisten an, sollten diese ihn freilassen, in der neuen Weltordnung ein gutes Wort für selbige bei \nameref{pers:colthubor} einzulegen.

    \subsubsection{Wissen/Informationen}
        Als parktizierendes und leitendes Mitglied des \hyperref[sons:kult-von-colthubor]{Kultes von Colthubor}, besitzt \emph{George Smith} ausführliche Kenntnisse über Riten und Prophezeiungen des Kultes. Unter anderem ist ihm bekannt, dass in kurzer Zeit, endlich wieder versucht wird den großen \nameref{pers:colthubor} selbst in diese Welt zu führen. In \emph{George Smith}s Augen würde dies zum Ausgleich sämtlicher Ungerechtigkeiten führen.

        \paragraph{Wissen ohne Probe}
            Bereitwillig erzählt \emph{George Smith} von \nameref{pers:colthubor} und seinen künftigen Taten. In seinem Übermut würde er auch über den Ablauf, sowie die notwendigen Vorkehrungen sprechen, jedoch würde er den Protagonisten nicht ganz den korrekten Ablauf beschreiben. Er zieht es eher vor Ihnen die falsche Dauer der verschiedenen Schritte zu sagen, damit das Ritual mit Sicherheit zu einem Erfolg führt.

            \begin{table}[H]
                \begin{tabularx}{\textwidth}{|l|X|}
                    \hline
                    Probe & Information \\ \hline
                    Überreden (50 erschwert) & \emph{George Smith} kann von seiner Überzeugung abgebracht werden. Wenn dies der Fall ist, ist er durchaus bereit den Protagonisten über die tatsächlichen Dauern des Rituals zu informieren. Es ist ihm jedoch nicht möglich den genauen Ort des Rituals zu nennen. Über den \nameref{items:talisman-von-kor} weiß \emph{George Smith} nichts konkretes, sondern nur dass er vor dem Zorn \hyperref[pers:colthubor]{Colthubors} schützen soll. Was dies genau bedeutet ist ihm jedoch unbekannt.\\ \hline
                    Verborgenes Erkennen & Den \nameref{items:talisman-von-kor}, welchen \emph{George Smith} zwar um den Hals hängen hat, aber versucht durch die rechten Hand und Arm zu verdecken. Über die Herkunft und die Bedeutung des Talismans schweigt sich \emph{George Smith} aus, da er sich ertappt fühlt. (\emph{Dies erleichtert eine Probe auf \emph{Überreden} um 20}.)\\ \hline
                    Verborgenes Erkennen (krit.) & Die Schnörkel und Symbole auf dem Armen scheinen einen gewissen Sinn zu ergeben. Sie erinnern Stark an ein Auge an machen Stellen. An anderen Stellen jedoch wie Tentakeln. Es scheint auch kryptische Dinge beschrieben oder \emph{gemalt} zu sein die sich von dem sonstigen Stil abgrenzen und Zahlen wie Buchstaben ergeben. Insgesamt wird ein Ablauf des Rituals näher beschrieben und die falschen Informationen werden widerlegt. Es wird also der korrekte Tag des Rituals und dessen Dauer erwähnt. \\ \hline
                \end{tabularx}
                \caption{Eine Liste von Informationen, welche die Protagonisten nach Proben von \emph{George Smith} erhalten können.}
            \end{table}

    \paragraph{Inventar:} \nameref{items:talisman-von-kor}

    \paragraph{Mögliche Erscheinungsorte:} \nameref{loc:gefaengnis}
    
