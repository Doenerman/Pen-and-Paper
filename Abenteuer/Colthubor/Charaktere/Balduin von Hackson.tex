\subsection{Baldiun von Hackson}
    \label{pers:baldiun-von-hackson}
    \emph{Baldiun von Hackson} ist das religiöse Oberhaupt des \hyperref[sons:kult-von-colthubor]{Kults von Colthubor}. Er hat sich schon in sehr jungen Jahren für den \hyperref[sons:kult-von-colthubor]{Kult} interessiert und hat bereits mit $15$ Jahren sein seine \hyperref[bla]{Beschneidung} (Integrationsritual) durchgeführt. Leider ist \emph{Baldiun von Hackson} nicht mit einem außergewöhnlichen Intellekt gesegnet. Auch wenn die Schriften Colthubors ziemlich explizit sind, ließ sich von anderen Kultmitgliedern \emph{Baldiun von Hackson} leicht manipulieren und als Marionette in der Öffentlichkeit ausnutzen.
    
    \subsubsection{Erscheinungsbild}
        \emph{Baldiun von Hackson} ist ein gut gekleideter, etwa 30 Jahre alter Mann, der sich die Haare rasiert, weil er das Gefühl hat es gäbe ihm einen müstischere Austrahlung. Grundsätzlich trägt er immer ein knall gelbes Gewand. Ihm ist nicht ganz klar warum, aber andere Kult mitglieder fanden schon immer, dass er darin herrlich bescheuert aussieht.

    \subsubsection{Verhalten}
        Aufgrund seines unterdurchschnittlichen Intellekts, versteht \emph{Baldiun von Hackson} nicht welches Übel sich in der Welt befindet. Dies führt jedoch dazu, dass er alle Menschen höchst zuvorkommend behandelt. Er bemüht sich stehts eines guten Umgangstons. Aufgrund seines mangelnden Charismas wirkt dies jedoch eher wie ein ungehörter Ruf nach Aufmerksamkeit. Allerdings zitierte er gerne aus den Schriften an willkürlichen Stellen.

        \paragraph{Zitate}
            Grundsätzlich ist die Idee, dass \emph{Balduin von Hackson} eine sehr oberflächliche Erkentniss aus den Zitaten zieht und nicht über tiefere Sinne nachdenkt. Daher sollen die Zitate zunächst positive und potentiell indoktirinierend klingen, aber subtil eigentlich darauf hinweisen, dass alles zu Grunde geht.
            \begin{itemize}
                \item \begin{displayquote}
                        "So werden die Früchte des Glaubens alle erreichen."
                      \end{displayquote}
                    Mit den Früchten des Glaubens ist eigentlich die Beschwörung von \nameref{pers:colthubor} gemeint. Mit anderen Worten, niemand wird von seiner Anwesenheit verschont.

                \item \begin{displayquote}
                        "So denn genug Glauben in der Welt ist, so wird \nameref{pers:colthubor} allen gleichermaßen helfen."
                      \end{displayquote}
                    \nameref{pers:colthubor} will offen gestanden niemandem helfen. Dies sagt in erster Linie also aus, dass \nameref{pers:colthubor} alle gleichermaßen bestrafen/Leid zufügen möchte.
                \item \begin{displayquote}
                        "Einem jedem was er verdient hat!"
                      \end{displayquote}
                \item \begin{displayquote}
                        "Zufall ist das gerechteste Schicksal."
                      \end{displayquote}
                      Dies ist eine Anspielung darauf, was passiert wenn \nameref{pers:colthubor} in diese Welt erscheint, dass die Menschen eine Wandlung erfahren werden und eine zufällige Fähigkeit erlangen werden.
            \end{itemize}


    \subsubsection{Wissen/Information}
        Auch wenn \emph{Baldiun von Hackson} das Oberhaupt der \hyperref[sons:kult-von-colthubor]{Kultes} ist, so weiß er doch erstaunlich wenig von ihm. Dieses Wissen ist aber sehr bereit zu teilen.

        \paragraph{Wissen ohne Probe} Abseit von den Orten \nameref{locs:bibliothek-der-alten-schriften} und \nameref{locs:zirkel-der-unbeschwertheit} der paar Zitate besitzt \emph{Balduin von Hackson} kein Wissen über den \nameref{sons:kult-von-colthubor}.
            

    \paragraph{Inventar:} -

    \paragraph{Mögliche Erscheinungsorte:} \nameref{locs:bibliothek-der-alten-schriften}, \nameref{locs:zirkel-der-unbeschwertheit}
