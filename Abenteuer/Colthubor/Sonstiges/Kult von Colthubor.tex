\subsection{Kult von \nameref{pers:colthubor}}
    \label{sons:kult-von-colthubor}
    Der Kult ist ein Zusammenschluss von Menschen, die an auf die Ankunft \hyperref[per:colthubor]{Colthubors} warten, diese gilt als Erlösung sämtlicher Ungerechtigkeiten und es heißt, dass jedem das widerfahren würde, was dem/derjenigen zustünde. Die Rituale/Riten dieser religiösen Gruppierung basieren auf Opferung, mit einem großen Schwerpunkt auf Blutopferung.

    \subsubsection{Beschwörungsritual}
        \label{sons:kult-von-colthubor:beschoerungsritual}
        Dieses Ritual muss ausgeübt werden, um \nameref{pers:colthubor} in seiner kompletten Form heraufzubeschwören. Wobei es weniger eine Beschwörung ist und viel mehr ein Portal, welches sich zu einer anderen Welt öffnet. Das Ritual ist grundsätzlich in drei Phasen eingeteilt.
        \begin{enumerate}
            \item Eingeleitet wird es durch ein großes Blutopfer. Entgegen sämtlicher andere Opfer gilt dies nur, wenn sich die ritualisierenden selbst Opfern. Dies muss nicht umbedingt das Leben kosten, aber es fordert einen hohen Tribut. Dieses Blut muss auf einem Bereich geschehen, der geprägt ist mit \emph{Runen}, die einen Kreis mit drei Schlangenlinien im gleichmäßigen Abstand haben.
            \item Sobald genug geopfert wurde, bildet sich in der Mitte des Bereiches eine große Blutlarche. Diese verfärbt sich in ein dunkles bis schwarzes Rot. Sollte sich ein anderes Blut als das eines Menschen in der Larche befinden, so öffnet sich das Portal nicht, bzw schließt sich wieder. In den \nameref{sons:kult-von-colthubor:alte-schriften} wird daher immer explizit von \emph{reiner} Opferung geschrieben.
            \item Wenn sich die Larche in voller Größe geformt hat, beginnt die \emph{Beschwörung}/der \emph{Übertritt}. Bei genauerem hingucken, ist es mögliche zu erkennen, dass die Larche wie ein Protal wirken muss, denn auf der anderen Seite sind Bewegungen zu erkennen (Probe \emph{Geistige Stabilität} ($1$ oder $2W8$)). Ein Effekt der Beschwörung ist die Metamorphose der Menschen. Diese Bedeutet, dass jeder Mensch eine der \emph{besondere} Eigenschaft/Fähigkeit erhält. Diese kann unterschiedlich schnell bemerkt werden.
        \end{enumerate}
        
        Die Protagonisten haben die Wahl, wie sie \nameref{pers:colthubor} besiegen können. zum einen, können sie das Ritual abbrechen, sie \nameref{pers:colthubor} bekämpfen oder sie können \nameref{pers:colthubor} \emph{verbannen}.
        \paragraph{Ritualabbruch}
            Das Ritual kann abgebrochen werden, in dem die Umstände für das Ritual verändert werden.Ein Ritualabbruch hat jedoch immer eine Entladung von Magie zur Folge. All jene, die sich über Runen befinden, erhalten $1W10$ Schaden, wogegen sich durch okulte Gegenstände geschützt werden kann. Dieser Schutz bedeutet, dass die Hälfte des Schadens absorbiert wird.
            \begin{itemize}
                \item \emph{Die Blutlarche.}
                    Die Blutlarche aus der zweiten Phase des Rituals kann \emph{verunreinigt} oder \emph{zerstört} werden.
                \item \emph{Die Runen.}
                    Die Symbole auf dem Boden können \emph{überschrieben} oder \emph{zerstört} werden. Die Runen sind jedoch nicht gewöhnlich in den Boden geschlagen. Um Sie umzuschreiben, muss entweder ein Opfer gebracht werden, oder ein okulter Gegenstand genutzt werden.
            \end{itemize}
        
        \paragraph{Eigentschaften/Fähigkeiten für Protagonisten}
            \begin{itemize}
                \item \emph{Ein loses Auge}. Dieses wird sofort bemerkt, wenn die Eigenschaft errungen wird. In dem Moment, die zweite Phase des \hyperref[sons:kult-von-colthubor:berschwoerungsritual]{Beschwörungsrituals} abgeschlossen ist, fällt dem Charakter das Auge raus. Dies geschiet komplett schmerzfrei. Das Auge selbst ist imun gegenüber sämtlichen Schaden und hat stehts eine Verbindung mit dem Sehsinn des Charakters.
                \item \emph{Erholsame Berührung}. Dies bedeutet, wenn dieser Charakter andere Menschen (Haut-zu-Haut) berührt, so fallen diese in einen unheimlich wohltuhenden Kurzschlaf (~ $5$ Minuten). Nach dem der berührte Mensch wieder aufwacht, sind die Hälfte der physischen Verletzungen geheilt. Der \emph{Schlafeffekt} wird bemerkt, bei erstem Kontakt, wobei der Erholungseffekt erst nach dem Aufwachen erst vermutet werden kann.
                \item \emph{Späte Einsicht}. Der Charakter mit dieser Eigenschaft, erhält nach abgeschlossenen Taten eine Erleuchtung, wie Dinge besser hätten gemacht werden können. Das Bewusstsein über diese Fähigkeit liegt bei den Protagonisten selbst, aber der erste Einsicht erhält der Charakter sobald das die zweite Phase des \hyperref[sons:kult-von-colthubor:beschwoerungsritual]{Beschwörungsrituals} abgeschlossen ist.
                \item \emph{Übermenschliche Muskeln}. Diese Eigenschaft ermöglicht es dem Charakter auf die sechsfache Körperkraft zuzugreifen. Es gilt hierbei zu beachten, dass es die Knochen von diese Eigenschaft nicht beeinflußt sind. Wenn also zu viel Kraft genutzt wird, so besteht die Gefahr dass Knochen brechen. Um dies durch eine Probe auf \emph{Geschicklichkeit} simuliert, wobei für jede nicht bestandene Probe am Stück, die nächste Probe um $5$ erleichtert wird. Diese Kraft fällt erst auf, wenn der Charakter mit dieser Eigenschaft Kraft oder Kondition auf irgendeine Weise nutzt.
            \end{itemize}

        \paragraph{Eigenschaften/Fähigkeiten der Anderer}
            \begin{itemize}
                \item \emph{Gedanken hören}. Der Charakter mit dieser Eigenschaft, hört die Gedanken der Menschen in seimem Umfeld, als würden der Mensch mit den Gedanken sie aussprechen. Gedanken der näheren Menschen sind somit lauter als die jener, die sich weit entfernt befinden. Diese Fähigkeit kann durch das Verschließen der Ohren ausgesetzt werden.
                \item \emph{Glück}. Diese Fähigkeit erleichtert jeden Wurf um $\frac{1}{10}$ der Würfelreichweite (Würfelgröße).
                \item \emph{Präzise Wetterkenntnisse}. Wer auch immer um diese Fähigkeit bereichert wird, ist in der Lage das Wetter stets korrekt vorher zusagen. Dies fällt jedoch erst auf, wenn der nächste Gedanke über Wetter auftritt.
                \item \emph{Zeitlupenbewegung}. Ein Charakter mit dieser Eigenschaft, bewegt sich ausschließlich extrem langsam. Dieser Effekt tritt ein, sobald die zweite Phase des Rituals abgeschlossen ist. Des Weiteren wirkt sich dieser Effekt auch auf das Reden aus, wodurch der Charakter langsam und mit einer tieferen Stimme spricht.
            \end{itemize}

    \subsubsection{Alte Schriften}
        \label{sons:kult-von-colthubor:alte-schriften}
        Die \emph{Alten Schriften} sind ein Sammelsurium an Steintafeln, einige in \emph{Bücherverbunden} zusammengeknoten, die den Ursprung und das ausleben des Glaubens an \nameref{pers:colthubor} beschreiben.

        \begin{itemize}
            \item \nameref{pers:colthubor} kann in diese Welt übertreten, wenn der Glaube an ihn stark genug ist.
            \item Glaube wird durch regelmäßige Opfer bewiesen, wobei gilt je größer das Opfer, desto größer gilt der bewiesene Glaube.
                \begin{displayquote}
                    Pflanzenopfer $<$ Opfer von persöhnlcher Bedeutung $<$ Tieropfer $<$ Menschenopfer
                \end{displayquote}
                Die Entstehung dieser Sortierung ist nicht in den Schriften dokumentiert.
            \item Es gibt eine Beschreibung von \nameref{pers:colthubor}, in der ein großer Fokus darauf gelegt wird, dass all jene verstummen, die im angesicht \hyperref[pers:colthubor]{Colthubors} Unglaube zeigen, verstummen werden.
            \item Es liegt eine genau Beschreibung vor, wie \nameref{pers:colthubor} in diese Welt \hyperref[sons:kult-von-colthubor]{überschreiten} kann. Zusätzlich gibt es eine grobe Einschätzung der Folgen des Rituals, etwas wie "\emph{Die Menschen erfahren nach einer Metamorphose ihr wahres Ich}".
        \end{itemize}

    Mitgliederliste: \nameref{pers:baldiun-von-hackson}, \nameref{pers:peter-klos}, \nameref{pers:george-smith}
