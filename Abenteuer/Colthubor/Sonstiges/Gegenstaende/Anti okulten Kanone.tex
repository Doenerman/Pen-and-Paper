\subsubsection{Anti okulten Kanone}
    \label{items:anti-okulten-kanone}
    Dies ist eine experimentelle Schusswaffe. Sie hat etwa die Größe eines halben Mensches und kann von einer Person nur erschwert genutzt werden. Es sieht aus, als bestünde sie aus einigen Röhren und Schläuchen. Sie besitzt eine Schublade, in die man okulte Gegenstände einfügen kann, welche während eines \emph{Schussprozesses} zerstört werden. 

    \paragraph{Nutzung}
        Die Kanone verfügt über einen Schuss. Sie kann zwar nachgeladen werden, aber nur mit okulten Gegenständen. Sollte sie mit einem \emph{gewöhnlichen} Gegenstand geladen werden, so führt es zu $40$\% zu einer Selbstzerstörung ($1W4$ Schade) und zu $60$\% zu einem Defekt ohne direkte Folgen für den Nutzer.

        Da es sich um eine experimentelle Waffe handelt, gibt es keine Gewähr, über den Effekt. Die Kanone ist grundsätzlich sehr schwer und deswegen für eine Person allein nur schwerer zu bedienen. Für eine präzise Handhabung, kann sie jedoch von mehreren Personen justiert werden, was von jedem eine Probe auf \emph{Geschicklichkeit} erfordert. Sollte diese gelingen, so besteht folgende Trefferprobe:
        \begin{displaymath}
            \frac{\emph{Zielattribut}-Distanz}{Personen} - 4 \cdot Geschicklichkeitspatzer
        \end{displaymath}
    Da es sich um eine experimentelle Waffe handelt, welche nicht ausreichend getestet wurde, besteht bei jedem Schuss die Möglichkeit, dass die Waffe unbrauchbar wird (\emph{Münzwurf}). Dieser erfolgt natürlich erst, wenn der Abzug bedient wird.
    
    \paragraph{Transport}
        Auf Grund ihres enormen Gewichtes, müssen die tragenden Charaktere insgesamt eine Körperkraft von $90$ oder mehr aufbringen. Bei allem drunter wird regelmäßig eine 10 minütige Pause benötigt, in welche die Waffe abgelegt wird. Es ist jedoch möglich eine Konstruktion zu bauen, welches die Waffe am Rücken tragen lässt und nur noch eine Körperkraft von $80$ benötigt wird.
