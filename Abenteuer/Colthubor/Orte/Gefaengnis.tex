\subsection{Das Gefängnis}
    \label{loc:gefaengnis}
    Das \emph{Gefängnis} befindet sich 20 $km$ außerhalb von der Stadt. Es befindet sich auf einer einzigen großen Lichtung. Von den Türmen aus kann man die nächsten 5 $km$ in alle Richtungen nichts als Gras sehen. Dies ermöglicht es der Wache im Falle eines Ausbruchsversuches die Mithilfe von außen früh zu erkennen und zu unterbinden, was vermutlich ein Grund dafür ist, dass in den letzten 17 Jahren nicht ein Gefangener lebendig aus dem \emph{Gefängis} fliehen konnte.

    Das \emph{Gefängnis} ist grundsätzlich aufgebaut wie eine Festung, eine rechteckige, etwa 10 $m$ hohe Mauer mit einigen Türmen zeigt den Inhaftierten ihre Grenzen auf. Die Mauer und Türme sind natürlich mit einigen Wächtern besetzt, welche alle wenigstens eine Schusswaffe tragen.

    Wenn die Protagonisten zu dem Gefangenen \nameref{pers:george-smith} gehen, kommen begegnen sie ebenfalls anderen Gefangen. Im vorbei gehen können Sprüche von
    \begin{quote}
        Geht es zu dem Typen der im Schlaf redet?
    \end{quote}
    von all jenen, die mit \nameref{pers:george-smith} noch keinen persönlichen Kontakt hatten oder noch relativ neu im \emph{Gefängnis} sind bis hin zu
    \begin{quote}
        Mit dem scheiß Freak und seinen Voodoo will ich nichts zu tun haben!
    \end{quote}
    von all jenen, die bereits lange genug inhaftiert sind, um von den grausamen \emph{Ritualen} die \nameref{pers:george-smith} auch im \emph{Gefängnis} ausübet, erfahren zu haben, gehört werden. Sollten einer der Protagonisten darauf eingehen auf den letzteren Kommentar eingehen wollen, so kann er von dem Gefangenen nicht nur erfahren, dass \nameref{pers:george-smith} den Rituale seine \hyperref[sons:kult-von-colthubor]{Kultes} nicht nur in Gefangenschaft immer noch nachgeht, sondern auch, dass er im Schlaf über diese und andere Prozeduren redet. Von seinem Zellennachbarn \nameref{pers:hubertus-bart} können sie auch noch konkreter erfahren was genau gesagt wird.


    \paragraph{Personenliste:} \nameref{pers:hubertus-bart}, \nameref{pers:george-smith}
