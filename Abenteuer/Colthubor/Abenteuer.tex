\documentclass[10pt]{article}
\usepackage{geometry}                % See geometry.pdf to learn the layout options. There are lots.
\usepackage{blindtext}
\usepackage[parfill]{parskip}    % Activate to begin paragraphs with an empty line rather than an indent
\usepackage{tikz}
\usetikzlibrary{arrows,automata,shadows,positioning,shapes}
\usepackage{graphicx}
\usepackage{amssymb}
\usepackage{amsmath}
\usepackage{amsthm}
\usepackage{epstopdf}
\usepackage{hyperref}
\usepackage{listings}
\usepackage{subfiles}
\usepackage[utf8]{inputenc}
\usepackage{float}
\usepackage{tikz}
\usepackage{tabularx}
\usepackage{graphicx}
\usepackage{caption}
\usepackage{wrapfig}

\begin{document}
    \tableofcontents

    % Informationen für die Spieler
    \newpage
    \input{"Vorab\space Information"}
    \newpage
    \section{Einleitung}
    Wir schreiben das Jahr 1903. Die Welt befindet sich in einem Zustand der Schwebe. Die Wissenschaft ringt mit der Relegion um die führende Macht der Gesellschaft. Ihr seit Mitglieder einer Organisation, die sich um den Fortbestand der Gesellschaft sorgt. Die \nameref{subsubsec:billo} (\emph{BiLLo}) versucht zwar nicht sonderlich geheim zu sein, dennoch ist sie nahezu unbekannt. Ihr werdet von denen belächelt die wissen für wen oder was ihr arbeitet und von all den anderen erntet ihr verwirrte Blicke. Mit dem Ausweis wedeln um Informationen zu erhalten klappt bei euch nur in den seltensten Fällen.

    Wie bei allen staatlichen Organisationen, seid auch Ihr von dem Fluch betroffen, dass nicht alles was ihr erlebt an die Öffentlichkeit gerät. Sonst hättet ihr einen ganz anderen Ruf. Es ist die \hyperref[subsubsec:billo]{BiLLo} gewesen, welche terroristische Organisationen wie \emph{die Jungs} nicht nur verhaftet, sondern auch wieder auf den richtigen Pfad gebracht haben. Heute zwingen die selbigen nicht mehr ihre Ideologie Folterei und Gewalt auf der Straße auf. Sie sind nun zivil und verbreiten ihre Botschaft nur noch musikalisch auf Hinterhöfen und der Straße. Aber all eure Verdienste in diesem Rahmen bleiben natürlich unpubliziert. Mit anderen Worten: \emph{Ihr kämpft für das gute, oder zumindest dafür, dass sich das schlechte nicht so schnell ausbreitet}.

    Leiter musste das \hyperref[subsubsec:billo]{BiLLo} vor einiger Zeit herbe Einbußen machen. Eure Hauptzentrale fiel einer Explosion zum Opfer. Dies ist das erste Mal, dass die \hyperref[subsubsec:billo]{BiLLo} groß in den Nachrichten stand, das erste Mal, dass die breite Öffentlichkeit ein Auge auf euch geworfen hat. Ganz besonders der Fakt, dass keiner der Mitarbeiter im Gebäude die Explosion überlebt hat warf große Fragen auf, die immer noch ungeklärt sind.


    \newpage
    % Informationen für den Spielleiter
    \section{Das Abenteuer}
    Das Abenteuer beginnt in einem Besprechungsraum. Die Spieler sitzen alle zusammen an einem Tisch. Ein relativ junger Mann steht vor euch und informiert euch über einen neuen Auftrage. Die Spieler erhalten folgende Informationen:
    \begin{itemize}
        \item Eine Sekte möchte die Gesellschaft ins Chaos stürzen. Sie plant das Machtgleichgewicht, welches sich über die letzten Dekaden entwickelt hat völlig um zustrukturieren. Das hart arbeitende Volk soll endlich den Lohn und Respekt bekommen, den es verdient hat.
        \item Im Gefängnis der \hyperref[subsubsec:billo]{BiLLo} befindet sich ein hochrangiges Mitglied der Sekte, \nameref{pers:george-smith}. \nameref{pers:george-smith} wurde schon vor einiger Zeit gefangen genommen, als selbiger, zusammen mit anderen Anhängern ein Blutopfer darbringen wollten. Die Anhänger sind während der Verhaftung oder des Rituals bereits gestorben.
        \item Die Sekte möchte in zwei Tagen das Gefüge der aktuellen Gesellschaftsstruktur überarbeiten.
    \end{itemize}


    \newpage
    \section{Orte}
        \label{sec:locs}
        Es folgt eine Liste von Orten, die im Laufe des Abenteuers erwähnt werden können und zu denen die Protagonisten erwartungsgemäß reisen können. Die Personen die bei den jeweiligen Orten auftauchen werden jeweils verlinkt. Zusätzlich wird natürlich erwähnt, was die Protagonisten entdecken können.
        \subsection{Gefängnis}
    \label{loc:gefaengnis}

    Personenliste: \nameref{pers:george-smith}


    \newpage
    \section{Ereignisse}
        \label{sec:events}
        In diesem Abschnitt gibt es eine Liste an verschiedenen Ereignisse, die der Spielleiter nutzen kann um die Protagonisten in gewisse Richtungen zu leiten. Die Ereigenisse sind unabhängig von dem was die Protagonisten.

    \newpage
    \section{Charaktere und Personen}
        In folgenden Abschnitt können alle Charaktere die bisher genannt wurden mit Beschreibung über das Verhalten und die vorhandenen Informationen gefunden werden.
    \subsection{Colthubor}
    \label{pers:colthubor}


    \input{"Charaktere/George\space Smith"}

    \section{Sonstiges}
    \input{"Sonstiges/Kult\space von\space Colthubor"}
\end{document}

