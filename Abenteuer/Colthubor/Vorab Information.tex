\section{Vorab Wissen}
    Dieses Wissen können die Spieler gerne vorab erhalten. Es enthält grundsätzliches Informationen über die aktuelle Situation in der sich die Welt in der die Charakter leben so wie die Instituation in der Sie arbeiten.

    \subsection{Die Welt}
        Die Welt befindet sich in einem Wandel. Die Zeit des Religiösität hat ihren Höhepunkt hinter sich und ein Zeitalter der Wissenschaft scheint bald anzubrechen. Die Dampfmaschine ist bereits erfunden und erfreut sich in den liberalen und neugirigen und natürlich reichen Kreisen der Gesellschaft großer Beliebtheit. In eben diesen Kreisen lassen sich auch vermehrt Rohre an Straßen und Häuserwänden finden, sowohl innen als auch außen, die nicht nur das Stadt\emph{bild}, sondern auch die Geräuschkullisse beeinflussen.

        \subsubsection{Die Gesellschaft}
            \label{subsubsec:die-gesellschaft}
            Das Symbol des Yin und Yang beschreibt die Gesellschaft im Staate Botianium. Auf der einen Seite gibt es den Wachsenden Keim der Wissenschaft, an denen sich überwiegend die Wohlhabenden bereichern. Sie profitieren mechanischen Pferden und Funken sprühenden Kochstellen die ganz ohne Feuer das Essen warm machen. Sie sind völlig eingenommen von dem Verständnis der Dampfmaschine und den neuen Möglichkeiten die sich daraus ergeben.

        \subsubsection{Vorherrschende Religionen}
            \label{subsubsec:vorherschende-religionen}
            All die jenigen, welche nicht offen für die neuen wissenschaftlichen Errungenschaften sind, halten oft an jenen Glauben fest an denen bereits auch ihre Eltern und Großeltern festgehalten haben. Der Glaube ist eine Säule, die stärkste und tragfesteste Säule eures Staates. Es bilden sich zwar kleine Risse, aber dennoch stüzt sie euer System. Es gibt allerdings zu viele Götter und Religionen um sie alle aufzuzählen. Es geht vom Antichristen bis hin zum Zoroastrismus und alles dazwischen. Die vorherrschende Religionen in Botianium, das Land in dem ihr euch befindet, ist der \nameref{par:potarismus} und die \nameref{par:stava}.

            \paragraph{Potarismus}
                \label{par:potarismus}
                Diese Religion ist aus der Philosophie entstanden, dass ein sorgenloser Geist zu einem sorgenlosen Leben führen. Traditionell wird versucht diesen Zustand über das Ritual des \emph{Fete} mit Speis und Trank zu erlangen. Es gibt jedoch nur wenige, welche das kleine Fenster der Unbekümmertheit dauerhaft halten können. Die jenigen die über das Ziel hinausschießen haben am morgen drauf oft zusätzliche Sorgen, wohingegen andere nicht weit genug zelebrieren um den Status der Sorglosigkeit überhaupt zu erreichen.

            \paragraph{Stäva}
                \label{par:stava}
                Eine Religion aus dem hohen Norden, welche die Brücke zwischen Glauben und Wissen zu schlagen versucht. Denn genau das ist es, was angestrebt wird. Ein Zustand in dem man alles zu Wissen vermag. Denn wer alles weiß, kann auch alles tun. So heißt es zumindest in den überlieferten Texten. Im allgemeinen stellt sich der Stävaismus streng gegen den Potarismus, da es offensichtlich ist, dass der Weg zur sorglosigkeit dazu führt Dinge auszublenden und zu vergessen, ein Sakrileg für all jene die an die Macht des Wissens glauben.



        \subsubsection{Bürgerliche interdisziplinäre Leit- und Lebensorganisation}
            \label{subsubsec:billo}
            Die \emph{Bürgerliche interdisziplinäre Leit- und Lebensorganisation} und ihre Angestellten versuchen den Status quo in der Gesellschaft so gut es geht zu erhalten. Diese Institution ist vom Staat organisiert aber eher unbekannt. Sie versucht in erster Linie die Gesellschaft und deren Interessen in geordneten Bahnen zu lenken, bzw den Wandel zu entschleunigen. Um dies zu tun, werden \emph{Querulanten}, wie sie intern genannt werden, von der Straße gezogen oder diskreditiert. Durch die Entschleunigung haben alle Parteien die Möglichkeiten sich vorzubereiten auf das was kommen mag. Es handelt sich also um die Bremse der Gesellschaft.
